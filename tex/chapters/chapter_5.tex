\section{Επίλογος}
\label{chapClose}
\subsection{Σύνοψη και Συμπεράσματα}

Η εργασία εκπονήθηκε μετά από συστηματική μελέτη του μαθηματικού υπόβαθρου της επιστήμης των πλεγμάτων, της κρυπτογραφίας δημοσίου κλειδιού και της ανάλυσης βέλτιστων αλγορίθμων παραλληλοποίησης και ταυτοχρονισμού. Χρησιμοποιήθηκε πληθώρα προγραμματιστικών εργαλείων, ενώ αναπτύχθηκαν και νέα εργαλεία για τις ανάγκες υλοποίησης και προσομοίωσης της λειτουργίας του συστήματος. Η συγγραφή του λογισμικού σε τρεις γλώσσες προγραμματισμού κάλυψε τις υψηλές απαιτήσεις που τέθηκαν εκ των προτέρων. 

Στόχος ήταν η επίλυση του προβλήματος του εγγύτερου διανύσματος, μέσω ενός συστήματος υψηλής απόδοσης, με ανοχή στα λάθη, πλήρως επεκτάσιμο και εύκολα διαμοιραζόμενο. Τα αποτελέσματα που προέκυψαν φανερώνουν πως η παράλληλη υλοποίηση των αλγορίθμων μπορεί να φέρει μεγέθη απόδοσης, εκθετικά μεγαλύτερα σε σύγκριση με τα γραμμικά συστήματα.  Το σύστημα ολοκληρώθηκε εκπληρώνοντας κάθε προσδοκία, όντας πλήρως λειτουργικό.

\subsection{Επεκτάσεις}

Η εργασία αυτή αποτελεί τον εναρκτήριο λίθο της ανάπτυξης ενός λογισμικού με επίκεντρο την τεκμηρίωση ασφαλείας ενός κρυπτογραφικού συστήματος βασιζόμενο σε πλέγματα. Τα επόμενα βήματα ανάπτυξης θα μπορούσαν να περιέχουν, την προσθήκη αλγορίθμων επίλυσης σε CUDA, που εκτελούνται σε κάρτα γραφικών, ώστε να αξιοποιείται πλήρως κάθε σύστημα. Επίσης την εφαρμογή τεχνικών βελτιστοποίησης σε αλγορίθμους που αναπτύχθηκαν. Επιπλέον τη δημιουργία ενός συστήματος παροχής υπηρεσιών ιστού μέσω του οποίου διευκολύνεται η συμμετοχή νέων κόμβων στο κατανεμημένο σύστημα και τελος τη διεξαγωγή πειραμάτων ευρείας κλίμακας υπό πραγματικές συνθήκες.