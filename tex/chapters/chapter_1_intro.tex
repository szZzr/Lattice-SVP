\section{Εισαγωγή}
\label{chapIntro}

\subsection{Προκλήσεις}

Η ασφάλεια αποτελούσε ανέκαθεν κύριο μέλημα στη ζωή κάθε ανθρώπου, σήμερα διανύοντας την εποχή της ψηφιοποίησης η ασφάλεια μας βασίζεται στη κρυπτογραφία, καθώς είναι αυτή που μπορεί δώσει λύση στα περισσότερα θέματα ασφάλειας ψηφιακών συστημάτων. Κάθε προσωπικό ή ιατρικό δεδομένο, κάθε ψηφιακή επικοινωνία ή ψηφιακό έγγραφο, οτιδήποτε συνδέεται στο διαδίκτυο είναι εκτεθειμένο, εάν δεν έχει κρυπτογραφηθεί. Παράλληλα, η διαρκώς αυξανόμενη επεξεργαστική ισχύς θέτει διαρκώς σε κίνδυνο την ασφάλεια των κρυπτογραφικών αλγορίθμων που διαθέτουμε,
 έτσι δημιουργείται η ανάγκη εύρεσης νέων αλγόριθμων και νέων τεχνικών κρυπτογράφησης.  

Τα σύγχρονα κρυπτογραφικά συστήματα αξιοποιούν δυσεπίλυτα μαθηματικά προβλήματα για την εφαρμογή των μεθόδων τους. Η έλευση των κβαντικών συστημάτων σηματοδοτεί την έναρξη της επίλυσης πολλών από αυτά, κάποια εκ των οποίων αποτελούν τη βάση των κρυπτογραφικών συστημάτων που χρησιμοποιούνται σήμερα. Έτσι δημιουργείται η ανάγκη εύρεσης προβλημάτων τα οποία εμφανίζουν ισοδύναμη πολυπλοκότητα επίλυσης, τόσο σε συμβατικούς ψηφιακούς επεξεργαστές όσο και σε κβαντικούς, ώστε να διατηρηθεί η ισότητα ανάμεσα στους χρήστες. Ένα τέτοιο πρόβλημα θεωρείται το πρόβλημα του μικρότερου διανύσματος σε πλέγματα (\lt Shortest Vector Problem) που χρησιμοποιείται στην Post Quantum Cryptography.

Η επίλυση του Shortest Vector Problem απασχολεί πληθώρα ερευνητών σε διεθνές επίπεδο αρκετές δεκαετίες. Τα τελευταία χρόνια έχει κεντρίσει το ενδιαφέρον της διεθνής κοινότητας κυβερνοασφάλειας, λόγω της εφαρμογής του προβλήματος στη κρυπτογραφία συστημάτων ανθεκτικών σε επιθέσεις με κβαντικούς υπολογιστές, με ανοιχτούς διαγωνισμούς επίλυσης του προβλήματος με συμμετοχές κορυφαίων επιστημόνων 
ώστε να εκτιμηθεί η αντοχή των αλγορίθμων που διέπουν το πρόβλημα. Η εφαρμογή αυτών των αλγορίθμων στα σύγχρονα κρυπτοσυστήματα παρέχει τη λύση στο πρόβλημα της ασφάλειας,
 κατά την ευρεία λειτουργία κβαντικών συστημάτων στο μέλλον.

\subsection{Συνεισφορά}
Στα πλαίσια της παρούσας πτυχιακής εργασίας, αναπτύσσεται μία προσπάθεια επίλυσης του Shortest Vector Problem παραλλάσοντας τον αλγόριθμο απαρίθμησης (enumeration algorithm) των Schnorr και Euchner, ο οποίος χρησιμοποιείται ως βασική υπορουτίνα στον αλγόριθμο αναγωγής BKZ. Η παραλλαγή του αλγορίθμου αναπτύχθηκε με κύριο μέλημα την εκμετάλλευση της επεξεργαστικής δύναμης 
 που παρέχει η εφαρμογή παράλληλων και κατανεμημένων αλγορίθμων. 

Η βασική ιδέα παραλληλοποίησης και κατανομής του προβλήματος στηρίζεται στο διαχωρισμό του δένδρου απαρίθμησης (enumeration tree) σε επιμέρους υποδένδρα αναζήτησης για τη εύρεση της λύσης του προβλήματος σ' αυτά, εφαρμόζοντας προηγμένες τεχνικές παράλληλου και ασύγχρονου προγραμματισμού. Διάφορες μελέτες\cite{conf/africacrypt/HermansSBVP10} που έχουν δημοσιευθεί στο παρελθόν, εφαρμόζουν τεχνικές παράλληλης επεξεργασίας με τη χρήση κάρτας γραφικών (GPU) σε περιβάλλον CUDA, ενώ εδώ μελετάτε η παράλληλη επεξεργασία στη κεντρική μονάδα επεξεργασίας (CPU) του συστήματος. Η αρχιτεκτονική του  κατανεμημένης συστήματος υλοποίησης δίνει επιπλέον τη δυνατότητα προσθαφαίρεσης κόμβων του δικτύου που συμμετέχουν στην επίλυση του προβλήματος ανεξαρτήτως φάσης της διεργασίας επίλυσης. 

Η επιλογή του αλγορίθμου απαρίθμησης έγινε με κριτήριο τη χαμηλή απαίτησή του σε επίπεδο μνήμης, σε σύγκριση με τους αλγόριθμους κοσκινίσματος (sieving algorithms), που ενώ υπερτερούν σε επίπεδο πολυπλοκότητας, έχουν εκθετικά μεγαλύτερες απαιτήσεις μνήμης. Το γεγονός αυτό, σε συνδυασμό με την ευκολία της προσθαφαίρεση κόμβων, συνεπάγεται στο ότι μπορεί να συμβάλει στην επίλυση του προβλήματος οποιοσδήποτε επεξεργαστής είναι συνδεδεμένος στο δίκτυο και έχει δυνατότητες παράλληλης επεξεργασίας, δεχόμενος εντολές από έναν κεντρικό κόμβο.

Στόχος της εργασίας, είναι η εξέταση την αντοχής του προβλήματος αυτού, εξετάζοντας ένα κατανεμημένο σενάριο επίθεσής του. Στο σενάριο που υλοποιείται στην εργασία, θεωρείται ένας  κεντρικός επιτιθέμενος κόμβος-σύστημα ο οποίος αφού διασπά το πρόβλημα σε επιμέρους προβλήματα, διαμοιράζει τα υποπροβλήματα στους κατανεμημένους κόμβους και αυτά με παράλληλη επεξεργασία αναζητούν την λύση του. Το δίκτυο αυτό μπορεί να θεωρηθεί ως ένα botnet network κατά το οποίο πολλά ανεξάρτητα κατανεμημένα συστήματα ελέγχονται από έναν κεντρικό σύστημα, υπακούοντας απομακρυσμένα στις εντολές του.

\subsection{Οργάνωση της εργασίας}
Η παρούσα πτυχιακή εργασία είναι οργανωμένη σε πέντε κεφάλαια. Στο κεφάλαιο \ref{chapBackground} παρουσιάζουμε το μαθηματικό υπόβαθρο γύρω από τα πλέγματα και τα προβλήματα που ορίζονται βάσει αυτών. Έπειτα στο \ref{chapSVP}o κεφάλαιο παρουσιάζεται αναλυτικότερα το πρόβλημα του συντομότερου διανύσματος και η μεθοδολογία επίλυσής του. Στο κεφάλαιο \ref{chapImplementation} έχουμε την περιγραφή του συστήματος που υλοποιήσαμε καθώς και την παρουσίαση των μετρήσεών μας. Τέλος στο κεφάλαιο \ref{chapClose} καταθέτουμε τα συμπεράσματά μας, συνοψίζουμε τη μελέτη που έχει λάβει χώρα και προτείνουμε πιθανές επεκτάσεις. 
